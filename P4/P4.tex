\documentclass[11pt]{article}
\usepackage[margin=1in]{geometry}
\usepackage{enumitem}

%opening
\title{CSCI 491, Project P4\\
		\small{Dataset Presentation Feedback}}
\author{Group 8\\
		\small{Tao Huang and James Soddy}}

\begin{document}

\maketitle

Our group presented our dataset, and some of our progress with it, to the class on
Tuesday. We were generally well received, but our presentation was certainly not
without problems. We got some good feedback from several of our audience, and
we will address their points here.

\section*{Audience Reception}

The audience seemed to stay well engaged, and we didn't get the impression that any
of them had failed to follow us. The questions we received at the end of the
presentation were mostly in regard to what exactly our figures represented and
what techniques we had used in creating them. These questions tie well into the
general feedback we received, and will be addressed later.

We felt the presentation went fairly smoothly, overall. Before our scheduled talk, we 
talked over the slides and
discussed what role each of us would play in the presentation, but we did not rehearse or
time our presentation. We had a few small problems which we feel could have been avoided
had we done a rehearsal. It may, for instance, have become clear that our figures
were not properly labeled during a practice explanation of them.

\section*{Shape of Feedback}

The overall evaluation sores of our presentation ranged from two to five. Four was the
median and the mode with four of the ten scored feedback forms showing that mark.
Five came in second with three received.

In particular, evaluators seemed to appreciate our interesting topic, concise 
explanations, presentation style and data visualizations. I think those are all
fair assessments. Our group was one of only three which kept our presentation
within the time limit, yet we managed to convey the bulk of our information within
that short time. Our topic is also a strength, as it is easily understood by an
audience unfamiliar with the specifics of our work, as well as being rather fun.

Some fairly common negative points on our evaluations were that we failed to
tie our work adequately to topology, that our visuals were not adequately labeled, that
the middle of our presentation had too many words per slide and too few figures, and that
we did not clearly express the goals of our project. We generally agree with these
criticisms, and this feedback will be very useful as we prepare our final project.

The biggest problem with the presentation, in our opinion, was our lack of proper
labels on our visualizations. To compound that problem, we discussed the figures in
a fairly hurried way without giving adequate explanation. The audience was able to
follow along with their general meaning, but questions following the talk, as well
as peer feedback, made it clear that more information was required. Our familiarity
with the data blinded us to the fact that it was not obvious to others what was
represented.

It is also true that some of our slides were fairly wordy. This is another issue
that a talk rehearsal might have pointed out to us. It would have been fairly easy to
mix in a few more visuals through this portion of the talk, and reduce the text
to main points. The wordy slides also led to some amount of robotic reading, which
was also pointed out by a few evaluators.

Our failure to clearly express our goals, or to explain how topology applies to
our project, are probably a result of how often these things have changed as
a result of problems we have run into. These issues will receive coverage in depth
with our next deliverable. We note, however, that our feedback was fairly
balanced on this topic. Some evaluators thought our goals and approaches were
clear and well motivated. These issues are so central to our work, however, that
we need to make certain that they are understood by everyone.

\section*{Conclusion}

Presenting our data to the class was a useful gauge of how our progress compares
to our classmates, and of what information will be useful and interesting in our
final presentation. Although there was quite a bit of room for improvement, we
feel good about our presentation and think that we succeeded in sharing our data
and methods with the class. We also clearly received the message that we need
a lot more progress before we are done.


\end{document}
