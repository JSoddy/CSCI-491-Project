\documentclass[11pt]{article}
\usepackage[margin=1in]{geometry}

%opening
\title{CSCI 491, Project P1}
\author{Tao Huang and James Soddy}

\begin{document}

\maketitle

Although our group had some trouble with coming up with ideas of what sort of project
to complete, in the end we came up with several which we think are worthy of
further consideration. However, we hope that we are not necessarily constrained in our 
final project choice by the ideas we will present here. We found that in gathering
information for our proposals, we learned quite a bit about what sorts of things
could make for interesting studies. Although we find these projects interesting, we
would like to also give some further consideration over the next week to other ideas we 
have.

\section*{English Letter Recognition}
Though there are only 26 letters in the English alphabet, individuals' handwriting 
can differ substantially. One letter, which is often seen as invariant, can have many
similar but different forms which pose a challenge to programs attempting to accept them 
as input. However, there are certain patterns involved in the writing of letters which
lend themselves to identifying them topolgoically. For instance, a 
"C" does not have a closed loop, which differentiates it from "O". These invariant differences
allow English letters to be placed in distinct groups. There are several distince
Homeomorphism groups which have been identified on English letters, as follows: (A,R),(B),(C,G,I,J,L,M,N,S,U,V,W,Z),(D,O),
(E,F,T,Y),(H,K),(P,Q) and (X)\cite{TopoIntro}. This means that the topological properties 
of the English letters can facilitate letter recognition. Letters can be quickly sorted into
their groups, simplifying the process of identifying them precisely. 

One possibility for this project would be to create a program which groups letters
based on their topology and uses those groupings to identify them. We would be able to 
generate our own data sets by simply writing in English, or collecting writing samples 
from people we know. If we found that the program was unable to differentiate some letters
consistently, we could attempt to find additional topological features which would
distinguish them.

Another option would be to attempt to not just distinguish one letter from another, but
to attempt to distinguish by whom a piece of writing was created. By taking a sampling of
the points from each letter in a writing sample, and building a Čech complex from them,
we would be able to find differences in writing samples based on when given features appeared
and disappeared. Minor differences in writing could cause cycles to appear and disappear
differently, and if we found that these patterns were consistent across writing samples
for an individual, then they could be used to uniquely identify that individual's writing.

\section*{Image Recognition}
For this concept, we would choose some distinct but similar objects and attempt to
differentiate them topologically. For instance, we might choose a hand and a foot.
We could convert images of the objects into simple black and white bitmaps, and then attempt to
identify their topological properties. We could start with simple, canonical,
representations of the objects, and after we had designed a method to differentiate
between those, we could challenge it with alternative views (i.e. a side view of a
foot or a closed hand).

The procedure for this could involve turning the pixels of an image into a connected
graph. We could bundle the points together based on their proximity to each other,
thus dividing the object into parts which should remain fairly constant even when
the object moves or is observed from a different angle.

By completing this project, we would hope to be able to find similarities between
seemingly different objects, as well as differentiate common objects which are often
seen as quite similar.

\section*{Topology of Poker Games}
Topology has long been used to study games. Many properties of games
lend themselves well to topological analysis, especially those which have 
natural topological properties (e.g. Hex). Games have been shown equivalent to
certain topological spaces, and been used to demonstrate topological
principles\cite{Cao02}\cite{Gale79}. Other games have been created by mathematicians
to be played in topological spaces, some of which have been useful in proving
certain mathematical ideas\cite{TopoGame}\cite{BanachGame}\cite{Kenderov93}.

Poker is an extremely complex game. There are over 2.5 million, ${52 \choose 5}$,
possible 5-card hands, and over $8\times10^{67}$, $52!$, possible card orderings
of a deck. In addition, most poker games allow for between 2 and 5 rounds of betting,
each of which can comprise up to 40 decision points, with each point allowing
at least 3, and up to an arbitrarily large number, of possible actions by participants. On
top of that, poker is a game of incomplete information. Any number of situations
with very different values may appear identical to players without knowledge
of their opponents' cards. It is true that some simpler forms of poker have been
solved in a limited sense\cite{Bowling15}. However, general poker is complex
enough that it is unlikely to be solved in a meaningful way any time soon.

We propose to take a large set of poker hands and perform topological analysis of
them. First, we would simply define a topology on a poker game, and look for any
shape that may arise from the data we have. Then we could proceed to use topology
to find which factors were the most significant (i.e. Is the shape of the data
changed substantially, or very little, as we change some factor). Using that
information, we may be able to create a simplified data set which still has
most or all of the topological properties of the original data.

Depending on our results from those efforts, we may be able to find an equivalence
between poker and some topological games, or other games which have been
studied with topology. Alternatively, we could use the
simplified data to demonstrate a general strategy which could be expected to
perform well at some form of poker. Or, if we see some interesting shape
emerge as we study the data, we may be able to demonstrate some property of
poker which has not been previously observed.

\section*{Conclusion}
Regardless of which direction we take our project, our first step will be to find
academic papers applicable to our domain. We will try to identify what has been done
in the area, as well as areas of open research which we may want to focus on. By the
time of our next deliverable, our main goal will be to have a good understanding of
our problem, a specific goal in mind for our results, and a time line for
our work.

\newpage
\begin{thebibliography}{9}

\bibitem{TopoIntro}
  https://en.wikipedia.org/wiki/Topology\#Introduction

\bibitem{Bowling15}
  Bowling, Michael, et al.,
  \emph{Heads-up limit hold'em poker is solved},
  Science,
  2015. 347: p 145-149.
  
\bibitem{Gale79}
  Gale, David
  \emph{The Game of HEX and the Brouwer Fixed-Point Theorem}
  American Mathematiccal Monthly,
  1979. 86: p 818-827
	
\bibitem{Cao02}
  Cao, Jiling, et al.,
  \emph{Topological Properties Defined by Games and Their Applications}
  Topology \& Its Applications,
  2002. 123(1): p 47-55.
  
\bibitem{Kenderov93}
  Kenderov, P.S., and J.P.Revalski
  \emph{Banach-Mazur Game and Generic Existence of Solutions to Optimization Problems}
  Proceedings of the American Mathematical Society
  1993. 118: p 911-917.
  
\bibitem{TopoGame}
  http://en.wikipedia.org/wiki/Topological\_game.

\bibitem{BanachGame}
  http://en.wikipedia.org/wiki/Banach-Mazur\_game.

	
\end{thebibliography}

\end{document}
