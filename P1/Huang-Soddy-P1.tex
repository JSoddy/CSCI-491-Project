\documentclass[11pt]{article}
\usepackage[margin=1in]{geometry}

%opening
\title{TODO: Title Here}
\author{TODO: Group Number Here\\ TODO: Group Names Here}

\begin{document}

\maketitle

TODO: brief introduction

\section{Direction 1}
 English Letter Recognition

Though there are only 26 letters in the English alphabet, people's handwriting of them 
can differ a lot. Different writings of one letter, which can be seen as an invariant, 
pose a challenge to a computer program which is supposed to accept standardized input. 
However, there are certain patterns involved in the writing of letters which may enable 
us to use the letter invariant to group different writings together. For instance, a 
"C" is not supposed to have a loop, or  it would look like "O". Such an observation, in 
fact, corresponds to part of topology concerning loops. Referring to Wiki, we find there 
are Homeomorphism groups of letter as follows: (A,R),(B),(C,G,I,J,L,M,N,S,U,V,W,Z),(D,O),
(E,F,T,Y),(H,K),(P,Q) and (X). This means that the topological properties of the English 
letters can facilitate letter recognition a lot, since humans follow the topological rule 
when writing and there are fewer candidates of letters to be recognized once the letters 
are categorized topologically. 

The data we use for this proposed project may be our own handwriting of letters. We may 
start writing letters by printing. When our program can recognize printed letters, we may 
proceed to test it on some more cursive letters. If the program can't distinguish 
between some handwritten letters, we group these letters and see what extra rules can be 
added to the program in order to endow it with more tolerance of human writing.   

As the input of a handwritten letter is a whole set of pixels, the first step connects 
the pixels to find if there's any loops. After this step, all handwritten letters are 
transformed into connected graphs, which then are put into different Homeomorphism 
groups. There are two Homeomorphism groups with only one element, so we can easily 
figure out who they are using a program. When dealing with other groups, we may 
subdivide the groups into groups including more similar items by applying some criteria 
that enables a program to distinguish between groups. There are lots of references 
and resources we can draw on to create criteria for our program, if we have a hard 
time figuring out the rules for the program. We may deepen the recognition program 
to be tolerant of cursive hand writing by integrating more rules in a concerted way, 
if time allows.



\section{Direction 2}
Distinguishing between a hand and a foot

The input image is a hand or a foot which will be translated into black and white 
pictures. we have pixels of an image after this. Topologically, the invariants are 
the hand and the foot, though, different angles of photo shooting produce different 
sets of pixels. Regardless of the shoot angle, the hand may take on different 
configurations which pose a challenge to a program that attempts to recognize it. 
We start with very "standard" images of feet or hands to see if the designed program 
distinguish between them. 

The first step involved is translating the image into a black and white picture. The 
technique required here can be easily found. Then it'll be  translating the pixels 
into a connected graph. To find a way to bin the data points takes some topological 
stuff such as the shape of the bin to be used, the distance between points, and the 
direction. As the invariants are certain, all we need to do is using different criteria 
to order to easily find out the invariants that represent the shape of a hand or a foot. 
Different shapes of bins may be tried, as well as different distances between pixels. 
If the program can achieve a connected graph we'll add some more rules to the program 
involving the numerical differences between two invariants. Then tests will be put on 
different input images to see how tolerant of different angles of photo shooting the 
program is. If this goal is achieved and if time allows, we may proceed to add more 
rules to the program to endow it with more tolerance of different configurations of 
hands and feet. 

\section{Discussion}
It is usually nice to conclude any write-up.

\end{document}
